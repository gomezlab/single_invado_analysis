%\documentclass[draft]{beamer}
\documentclass{beamer}

\mode<presentation>
{
  %\usetheme[secheader]{Boadilla}
  \usetheme{default}
}

\usepackage{calc}
%\usepackage{multimedia}
\usepackage{movie15}

\usepackage[english]{babel}

\usepackage[latin1]{inputenc}

\usepackage{times}
\usepackage[T1]{fontenc}

\newcommand{\figref}[1]{\begin{flushright}{\tiny #1 }\end{flushright}}
\newcommand{\centerfig}[3]{
	\begin{center}
	\begin{figure}[htbp]
	\includegraphics[#1]{#2}
	\\{\tiny #3}
	\end{figure}
	\end{center}
}

\title[]{High Content Analysis of Invadopodia Imaging}

\author[] % (optional, use only with lots of authors)
{Matthew Berginski}
% - Use the \inst{?} command only if the authors have different
%   affiliation.
  
\date[Lab Meeting 7/7/2010]{Bear Lab Meeting 7/7/2010}

\usecolortheme[RGB={2,91,173}]{structure}

\begin{document}

\begin{frame}
	\titlepage
\end{frame}

\begin{frame}
	\frametitle{Invadopodia Degrade the Extracellular Matrix}
	\begin{columns}
		\begin{column}{0.5\textwidth}
		\begin{itemize}
		\item For metastasis to occur, the extracellular matrix must be degraded
		\item Dynamic intra-cellular structures that contain many proteins
		\end{itemize}
		\end{column}
		
		\begin{column}{0.5\textwidth}
		\centerfig{width=\textwidth}{figures/invado_sketch.png}{Ayala, et al.
		Euro J of Cell Bio, 2006}
		\end{column}
	\end{columns}
\end{frame}

\begin{frame}
	\frametitle{Using Microscopy to Quantify Invadopodia}
	\begin{columns}
		\begin{column}{0.5\textwidth}
		\begin{itemize}
		\item Primary method to watch invasion uses collagen matrix mixed with
		fluorescent dye
		\item As invadopodia form and matrix broken down, dye diffuses away
		\end{itemize}
		\end{column}
		
		\begin{column}{0.5\textwidth}	
		\centerfig{height=0.4\textheight}{figures/sample_gel}{}
		\centerfig{height=0.4\textheight}{figures/sample_puncta}{}
		\end{column}
	\end{columns}
\end{frame}

\begin{frame}
	\frametitle{Prior Quantification Methods are Low Throughput}
	\begin{columns}
		\begin{column}{0.5\textwidth}
		\begin{itemize}
		\item Reliant on by hand counting
		\item number of cells with invadopodia
		\item number of invadopodia per cell
		\item tracking degradation over time
		\end{itemize}
		\end{column}
		
		\begin{column}{0.5\textwidth}
		\centerfig{width=\textwidth}{figures/invado_over_time}{Oser et al., J.
		Cell Biol., 2009}
		\end{column}
	\end{columns}
\end{frame}

\begin{frame}
	\frametitle{Image Processing Overview}
	\begin{enumerate}
	\item register images
	\item find the bright actin puncta
	\item track the puncta
	\item classify puncta as invadopodia or not
	\item analyze the data associated with the invadopodia
	\end{enumerate}
\end{frame}

\begin{frame}
	\frametitle{Dealing with Drastic Diagonal Drift in Demented Daguerreotypes}
	\begin{columns}
		\begin{column}{0.5\textwidth}
		\begin{itemize}
		\item major problem for the tracking algorithm
		\item registration and appropriate methods to deal with shifted images
		needed
		\end{itemize}
		\end{column}
		
		\begin{column}{0.5\textwidth}
		\centerfig{height=0.75\textheight}{figures/cascade_diagram}{}
		\end{column}
	\end{columns}
\end{frame}

\begin{frame}
	\frametitle{Finding the Actin Puncta}
	\begin{itemize}
	\item based on a method developed to identify focal adhesions
	\item uses a high-pass filter to identify puncta in differing background fluorescence levels
	\end{itemize}
	\centerfig{width=\textwidth}{figures/invado_finding/finding_composite}{}
\end{frame}
		
\begin{frame}
	\frametitle{Tracking the Puncta}
	\begin{columns}
		\begin{column}{0.5\textwidth}
		\begin{itemize}
		\item system relies on overlap in puncta from frame to frame
		\item if overlap fails, uses distance between puncta centroids
		\end{itemize}
		\end{column}
		
		\begin{column}{0.5\textwidth}
		\centerfig{width=\textwidth}{figures/tracking_flowchart}{}
		\end{column}
	\end{columns}
\end{frame}

\begin{frame}
	\frametitle{Filtering to Find Invadopodia}
	\begin{columns}
		\begin{column}{0.5\textwidth}
		\begin{itemize}
		\item longevity
		\item doesn't merge with another puncta
		\item didn't split off from another puncta
		\item decrease in gelatin underneath puncta
			\begin{itemize}
			\item local background difference
			\end{itemize}
		\end{itemize}
		\end{column}
		
		\begin{column}{0.5\textwidth}
		\centerfig{width=0.9\textwidth}{figures/Local_background}{}
		\end{column}
	\end{columns}
\end{frame}

\begin{frame}
	\frametitle{Overall Local Difference Filtering Cartoon}
	\centerfig{width=\textwidth}{figures/sm_puncta/local_diff_filtering_samp50}{}
\end{frame}

\begin{frame}
	\frametitle{Real Puncta Time-lapse}
	\centerfig{width=\textwidth}{figures/sm_puncta/montage_composite}{}
\end{frame}

\begin{frame}
	\frametitle{Quantifying the Local Difference Measurement}
	\centerfig{height=0.8\textheight}{figures/sm_puncta/ctrl_02_23_pos03_0515}{}
\end{frame}

\begin{frame}
	\frametitle{Visualizing the Results - Highlighting Invadopodia}
	\centerfig{height=0.7\textheight}{figures/final_vis/invado_highlight_last_no_high}{}
\end{frame}

\begin{frame}
	\frametitle{Visualizing the Results - Highlighting Invadopodia}
	\centerfig{height=0.7\textheight}{figures/final_vis/invado_highlight_last_highlights}{}
\end{frame}

\begin{frame}
	\frametitle{Data Overview}
	\begin{itemize}
	\item with the puncta tracked and classified, we can start looking at invadopodia properties
	\item working with a set of knockdowns
	\end{itemize}
\end{frame}

\begin{frame}
	\frametitle{Longevity Comparisons}
	\centerfig{width=\textwidth}{figures/data/longevity_barplots}{}
\end{frame}

\begin{frame}
	\frametitle{Mean Puncta Area Comparisons}
	\centerfig{width=\textwidth}{figures/data/area_barplots}{}
\end{frame}

\begin{frame}
	\frametitle{Mean Local Difference Comparisons}
	\centerfig{width=\textwidth}{figures/data/local_diff_means}{}
\end{frame}

\begin{frame}
	\frametitle{Future Plans}
	\begin{itemize}
	\item work on last few knockdowns
	\item track down false positives in BB94 cells
	\item add more invadopodia property comparisons (e.g. - distance from edge)
	\item implement invadopodia producing cells counting software
	\end{itemize}
\end{frame}

\end{document}
