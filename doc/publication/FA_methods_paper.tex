\documentclass[letterpaper,twocolumn]{article}

\pdfpagewidth=\paperwidth
\pdfpageheight=\paperheight

%setup double spacing for editing
\usepackage{setspace}
%\doublespacing

\usepackage{graphicx} %get graphics commands
\usepackage{psfrag}
\usepackage{times}

\title{Automated Identification and Tracking of Focal Adhesions in Fluorescent TIRF Images}

%get \FloatBarrier command
\usepackage{placeins} 

%give option to use commands like 0.5\textwidth for distances
\usepackage{calc} 
 %allows wraping figures around text
\usepackage{wrapfig}

%adds option to include verbatim input from files
\usepackage{moreverb} 

\begin{document}

\maketitle

\begin{abstract}

\end{abstract}

\section*{Introduction}

\section*{Methods}

Several stages are needed to quantitatively examine the FA from a single experiment. Following experimental data collection, the images were analyzed to locate, track and quantify several different properties of each FA.

\subsection*{Image Processing}

\begin{figure}[htbp]
\begin{center}
\includegraphics[width=0.5\textwidth]{figures/place_holder}
\caption{Results of image processing steps which identify focal adhesions, here outlined in green.}
\label{cell_sample}
\end{center}
\end{figure}

Methods to identify the FA were adapted from a prior publication \cite{Zamir1999}, with some specific modification. Briefly, each image taken during an experiment was high pass filtered, using a round averaging filter with a radius of 11 pixels (4.7 $\mu$m diameter). To identify a threshold (thresh$_{fa}$) was determined empirically and set for all processed images (0.1 on images with pixels normalized to between 0 and 1). The \emph{water} segmentation method was used as described, but with the following modifications. When a pixel acts as bridge between to large adhesions, where large is defined as 40 or more pixels (1.85 $\mu$m$^2$), the bridge pixel is assigned to the adhesion whose centroid is closest to the bridge pixel. Also, following addition of all pixels whose high passed image value was above thresh$_{fa}$, hole pixels were identified and added to adhesion map using the same \emph{water} algorithm. Between 200 and 600 adhesions were found in each image from the experimental data. With the focal adhesion locations identified in each cell, the properties of each FA in the experimental data set were collected including: 

\begin{itemize}
\item size
\item centroid position
\item average pixel intensity in each adhesion 
\item distance from the cell edge to FA centroid
\item distance from the cell centroid to FA centroid
\item major/minor axes lengths
\item solidity 
\end{itemize}

\subsection*{FA Tracking and Analysis}

With the focal adhesions identified in each image of the experimental data set, another series of algorithms were designed to track the focal adhesions through each sequential image. The tracking algorithm is based on a birth-death model of a FA lifetime. In each sequential image a FA can either be born, continue into the next time step, merge or die. The birth-death-merge processes are detected by examining the properties extracted from the segmented adhesions. The results of this tracking algorithm are assignments of the FA identified in each image into lineages that track the development of the FA during the course of the experiment.

\begin{figure}[htbp]
\begin{center}
\includegraphics[width=0.5\textwidth]{figures/fa_workflow}
\caption{Workflow of adhesion tracking}
\label{tracking_chart}
\end{center}
\end{figure}

The tracking algorithm is initialized with all the adhesions detected in the first frame of the image sequence (Figure \ref{tracking_chart}). The first step of the tracking algorithm attempts to locate FA which correspond in the next time step of the experimental data. This first step assumes that if a focal adhesion overlaps with a focal adhesion in the following frame, that these overlapping adhesions correspond to one another. When an adhesion overlaps with more than one adhesion in the following frame, the adhesion with the greatest percentage of overlap is assigned as the match in the next frame. If a FA does not overlap with any of the FA in the following image, the FA closest to that adhesion in terms of the euclidian distance between each adhesion's centroid is assigned as a match. All of the living focal adhesions are assigned a corresponding FA in the following image by these two rules.

This process of assigning live adhesions to corresponding adhesions in the following frame produces sets of adhesions that are predicted to merge. Some of these merge events are true merge events where one adhesion has joined with another, while others are adhesions which die, but are erroneously assigned as merge events. When a FA does not overlap with the FA it is predicted to become, this FA is assumed to have died and its lineage is ended. For the remaining merge event where more than one adhesion has been predicted to merge in the next frame, one of the merging FA lineage is selected to continue, while the other FA lineage is predicted to end. When the adhesions predicted to merge differ in size by at least 10\%, the larger adhesion's lineage is continued. If the merging FA's sizes do not differ by at least 10\%, the lineage whose current centroid is closer to the adhesion centroid in the following image is predicted to continue. By this sequences of rules, each merge event is resolved so that corresponding FA in adjacent experimental data images are collected.

Following tracking live adhesions and resolving the merge and death events, there remain FA in the following image which are not assigned to any of the current lineages. The unassociated FA are assigned into new lineages. This process of tracking the live adhesions, resolving merge and death events and starting new lineages is repeated for each image in the experimental data sequence. 

The tracked FA lineage allow the properties of each adhesion %Pick up here%

With the FA lineages collected various properties of the FA lineage can be extracted. These include:

\begin{itemize}
\item average pixel intensity during lifetime
\item average speed
\item longevity
\end{itemize}

\subsection*{Accumulation and Decay Estimates}

To estimate the rates of Paxillin accumulation and decay during the life cycle of a FA, an automated method to identify the slopes of Paxillin intensity versus time curves was developed. 

\section*{Results}

\begin{figure}[htbp]
\begin{center}
\includegraphics[width=0.4\textwidth]{figures/area_vs_dist}
\caption{Statics Plots}
\label{statics_figure}
\end{center}
\end{figure}

\begin{figure}[htbp]
\begin{center}
\includegraphics[width=0.4\textwidth]{figures/place_holder}
\caption{Dynamics Plots}
\label{dynamics_figure}
\end{center}
\end{figure}

\begin{figure}[htbp]
\begin{center}
\includegraphics[width=0.4\textwidth]{figures/place_holder}
\caption{Accumulation + Decay plots}
\label{accum_decay}
\end{center}
\end{figure}

\section*{Conclusion}

\bibliography{all_literature}{}
\bibliographystyle{plain}

\end{document}