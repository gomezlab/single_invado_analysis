\documentclass[letterpaper]{article}

\pdfpagewidth=\paperwidth
\pdfpageheight=\paperheight

\usepackage{graphicx} %get graphics commands
\usepackage{times}

%get \FloatBarrier command
\usepackage{placeins} 

%give option to use commands like 0.5\textwidth for distances
\usepackage{calc} 
 %allows wraping figures around text
\usepackage{wrapfig}

%adds option to include verbatim input from files
\usepackage{moreverb} 

\usepackage{anysize}
\marginsize{1in}{1in}{0.45in}{0.45in}

\title{Data Set Description}
\author{Matthew Berginski}
\begin{document}

\maketitle

The formatting on the email I was writing was becoming a little difficult, so here is a more readable description of the contents of the forwarded data set. Any italicized word refers to a file name.

%%%%%%%%%%%%%%%%%%%%%%%%%%%%%%%%%%%%%%
\section*{\emph{individual\_pictures}}
Contains zero padded numbered folders, one for each image used in the analysis, there are missing numbers due to some images being out of focus and not usable. Each folder contains the following image files and \emph{raw\_data} directories.
\subsection*{Image Files}
	\begin{itemize}
	\item \emph{focal\_image.png}: The raw fluorescent Paxillin images, these are 12 bit images stored in 16 bit png files
	\item \emph{mRFP.png}: The raw myristolated RFP images, these are images used to find the cell edges, also they are 12 bit images stored in 16 bit png files
	\item \emph{adhesions.png}: binary image of the location of the focal adhesions in each image
       \item \emph{cell\_mask.png}: binary image of the location of the pixels inside the cell body
	\end{itemize}

\subsection*{\emph{raw\_data}}
A directory holding files with the properties of each adhesion, the files are comma separated value formated, most of the properties are those produced by the matlab command 'regionprops' and named according to the name used by regionprops. There are several other files in 
	\begin{itemize}
       \item \emph{Angle\_to\_center.csv}: the angle from the center of the cell to the centroid of the adhesion
       \item \emph{Average\_adhesion\_signal.csv}: the average fluorescence value of the adhesion pixels
       \item \emph{Centroid\_dist\_from\_edge.csv}: how far the centroid of the adhesion is from the edge of the cell
       \item \emph{Variance\_adhesion\_signal.csv}: the variance of the fluorescence value of the adhesion pixels
	\end{itemize}

%%%%%%%%%%%%%%%%%%%%%%%%%%%%%%%%%%%%%%
\section*{\emph{adhesion\_props}}
This folder contains data files summarizing several sets of properties concerning the static adhesions, the pixel values of the experimental data and the focal adhesion lineages. The properties in these files are all in physical units, the base unit for distance is the micron, the base unit for area is the square micron and the base unit for time in the minute. All the files are formated to be friendly to R.

\begin{itemize}
\item \emph{individual\_adhesions.csv}: contains a few properties from each of the focal adhesions found in each image of the experimental data, properties include:
	\begin{itemize}
	\item I\_num - the image number where the adhesion appears
	\item ad\_num - the unique adhesion ID number
	\item Area - the number of pixels in the adhesion
	\item Average\_adhesion\_signal - the average value of the adhesion pixels
	\item Centroid\_dist\_from\_edge - the distance between the adhesion centroid and the cell edge
	\item Variance\_adhesion\_signal - the variance of the values in the adhesion pixels
	\end{itemize}
\item \emph{single\_lin.csv}: contains properties collected for each lineages, some properties can not be calculated and are identified as 'NaN' (not a number)
	\begin{itemize}
	\item longevity: number of a frames a lineage is alive
	\item largest\_area: the largest focal adhesion area
	\item s\_dist\_from\_edge: the distance from the cell edge at birth
	\item speed: the average speed throughout the adhesion lifetime
	\item max\_speed: the highest speed of adhesion centroid movement between any two frames
	\item ad\_sig: the average pixel value of all the pixels in the lineage
	\end{itemize}
\item \emph{pixel\_props.csv}: contains properties collected concerning the actual pixel values, used to produce plots to investigate photobleaching
\item \emph{plots}: a folder containing various graphs relating properties in the above files, R was used to produce the pdf versions, while imagemagick was used to convert the pdf files to png files
\end{itemize}


%%%%%%%%%%%%%%%%%%%%%%%%%%%%%%%%%%%%%%
\section*{\emph{tracking\_seq.csv}}

This is the matrix produced by the tracking algorithm, it specifies all of the lineages detected in the experiment. This file encodes all the detected lineages into a matrix, where each row is a single lineage and each column is a single image. A single row, which describes a focal adhesion lineage, consists of a string of focal adhesion ID numbers. The focal adhesion ID numbers uniquely identify a focal adhesion in a given frame, but are not unique across frames. These numbers were assigned by the matlab function 'bwlabel', using the image stored in \emph{adhesions.png} and the 4-neighbor connectively parameter option (see 'help bwlabel' for more information). 

Note that the ID numbers assigned by matlab are between one and the number of adhesions, but that all the numbers in the tracking matrix are offset downward by one. This is the result of the tracking algorithm being written in perl (array addressing in perl starts at zero). If you need to use the tracking matrix in matlab, remember to add one after loading the file.

Negative numbers in the tracking matrix indicate adhesions that have not been born yet or those that have died.

%%%%%%%%%%%%%%%%%%%%%%%%%%%%%%%%%%%%%%
\section*{Movie Files}
Several movies are also included with the data set

\begin{itemize}
\item \emph{orig\_track.mov}: The left frame displays the experimental data, while the right frame displays the lineages. Each lineage receives a color that is unique during the lifetime of the adhesion and the color remains with the adhesion during the entire lineage lifetime. When a lineage is born, the entire adhesion is filled in with the lineage color on the frame where it is born.

\item \emph{edge\_track.mov}: The left frame displays the edge of the cell and the edge of each focal adhesion, using a color unique to a time step. The cell and adhesion edges are drawn over the last image in later time steps. The right panel is the same as in \emph{orig\_track.mov}.

\item \emph{time\_track.mov}: The left frame is the same as in \emph{edge\_track.mov}. The right frame display the focal adhesions, where the edge color is determined by when the adhesion is born. Each time step has a unique color and each lineage maintains the same color throughout the movie.
\end{itemize}
\end{document}